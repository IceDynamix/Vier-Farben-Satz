\section{Vier-Farben-Satz}
\subsection{Beschreibung}
Der Vier-Farben-Satz ist komplex, viel komplexer als der Beweis vom Sechs- und Fünf-Farben-Satz, er ist aber ähnlich aufgebaut. Ihn zu beweisen würde den Rahmen dieser Arbeit sprengen. Es wurde schon gegen Ende 19. Jahrhunderts versucht ihn zu beweisen, oftmals aber leider ein ungültiger Beweis mit Fehlern und Gegenbeispielen \cite{wrong_proof}. Er wurde schlussendlich in 1976 von Kenneth Appel und Wolfgang Haken bewiesen. Der Vier-Farben-Satz war außerdem auch der erste Satz, der mit einem Computer bewiesen wurde, was aber damals kritisiert und hinterfragt wurde \cite{implications}. Danach wurde er in 1997 nochmal einfacher bewiesen, basiert aber trotzdem noch auf Computern \cite{robertson_new-proof}. Außerdem wurde er in 2005 mit Hilfe eines mathematischen Beweisassistenten bewiesen \cite{formal_proof}. \textit{``The Journey of the Four Colour Theorem Through Time``} von Andreea S. Calude bietet außerdem einen tollen Einblick um mehr Details über die Geschichte des Vier-Farben-Satzes zu bekommen. \cite{journey_through_time}. Sobald man den Vier-Farben-Satz bewiesen hat, kann man auch die effizienteste Methode finden um alles in vier Farben einzufärben \cite{efficiently_coloring}.

\subsection{Alltag}
Wie kann der Vier-Farben-Satz für den Alltag nützlich werden? Ein Beispiel wäre Platzierung von Funktürmen, die aufgrund von technischen Gründen nicht mit der gleichen Frequenz nebeneinander stehen dürfen. Wenn sie zu nah stehen, dann überlappen sich die Funkbereiche und ein Turm muss eine andere Frequenz nutzen. Das Verwalten von Funkfrequenzen kostet Geld, daher wird versucht so wenige Frequenzen wie möglich zu nutzen. In der Graphentheorie benutzt man Knoten für Türme, Verbindungen falls sich die Funkbereiche der Türme überlappen und Farben für die verschiedenen Frequenzen. Dadurch, dass sich alles auf einen planaren Graphen darstellen lassen lässt, kann man davon ausgehen dass nicht mehr als vier Frequenzen genutzt werden müssen. Generell lässt eine Anwendung des Vier-Farben-Satzes finden, solange man die Ausgangssituation auf einen planaren Graphen übertragen kann.