\section{Sechs-Farben-Satz}
Nur weil bewiesen wurde, dass es auf einem planaren Graphen einen Knotenpunkt mit weniger als sechs Verbindungen gibt, heißt das nicht, dass alles mit sechs Farben einfärbbar ist.
\subsection{Beweis}
\begin{theorem}
    Jeder planare Graph lässt sich mit sechs Farben oder weniger einfärben, ohne dass zwei Knotenpunkte gleicher Farbe verbunden sind.
\end{theorem}
\begin{proof}
    Anders als zum vorherigen Beweis basiert dieser nicht auf einen Beweis durch Widerspruch, sondern durch Induktion. Das ist ein fundamental anderer Ansatz. Einen Beweis durch Induktion kann man sich als \textit{``rekursive``} Bedingung vorstellen, die wie Dominos nacheinander umfallen. Dieser besteht aus zwei Bestandteilen. Zuerst muss bewiesen werden, dass eine Eigenschaft für eine natürliche Zahl $\mathbb{N}$ $(0,1,2,3\dots)$ gilt (Induktionsanfang). Dann muss bewiesen werden, dass, falls sie für eine natürliche Zahl $n$ gilt, sie auch für die nächste natürliche Zahl $n+1$ gilt (Induktionsschritt). Angenommen wir haben einen Graph mit $k$ Knoten. Falls $k\leq6$, dann ist das Einfärben einfach, da jedem Knoten eine Farbe zugewiesen werden kann (Induktionsanfang). Falls aber $k>6$, dann kann man nicht mehr einfach so Farben zuweisen. Das Ziel ist es jetzt unseren Graphen so zu verändern um auf die Situation $k\leq6$ kommen. Wie bereits bewiesen, gibt es immer einen Knotenpunkt der weniger als sechs Nachbarn besitzt. Wenn ein Knotenpunkt mit fünf Nachbarn vorliegt, dann ist dieser Knotenpunkt und dessen fünf Nachbarn immer mit sechs Farben einfärbbar, indem man fünf Farben für die Nachbarn und eine sechste Farbe für den Knotenpunkt selbst nutzt. Dieser hat für uns erst mal keinen Belang, es steht uns frei ihn zu entfernen, denn es wurde gezeigt, dass er im Nachhinein korrekt einfärbbar ist. Nun hat man einen neuen planaren Graphen ohne den entfernten Knotenpunkt, der wiederum auch einen Knoten mit weniger als 6 Nachbarn besitzen muss. Damit befindet man sich in der gleichen Ausgangssituation wie vorher, nur mit $k-1$. Das bedeutet, dass dies mit Hilfe der Induktion so weit fortgeführt werden kann, solange es noch Knoten zu entfernen gibt. Man wiederholt dies solange, bis man an $k=6$ angekommen ist und weist dann jedem Knoten eine Farbe zu. Nun werden die Knoten in der umgekehrten Reihenfolge wieder an den Graphen angefügt. Da es immer ein Knoten war, der weniger als 6 Nachbarn hatte, kann er daher immer mit einer sechsten Farbe eingefärbt werden, mit der er nicht umgeben ist. Dies wird so lange wiederholt, bis man beim ursprünglichen $k$ angekommen ist. Damit ist jeder Knoten eingefärbt.
\end{proof}